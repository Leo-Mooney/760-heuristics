
\documentclass{article}
\usepackage{graphicx} % Required for inserting images
\usepackage{amsmath}
\usepackage{algorithm}
\usepackage{algorithmic}

\title{760-Heuristics-Assignment}
\author{Leo Mooney}
\date{March 2023}

\begin{document}

\maketitle

\section{Question 1}
Neighbors of the current solution can be found by swapping one pot in
any crucible with a pot in any other crucible.
This can be formally defined as following:

$N(\mathbf{x}) = \{\mathbf{y}(\mathbf{x}, k, l, m, n), k=1,2,3,...,16, 
l=k+1,k+2,k+3,...17, m=1,2,3, n=1,2,3\}$ where

$$\mathbf{y}(\mathbf{x},k,l,m,n)=(y_{1, 1},y_{1, 2},y_{1, 3};
y_{2, 1}y_{2, 2}y_{2, 3};...y_{c, j}),y_{c,j}=
\begin{cases}
    x_{l, n} & \text{if } c=k,j=m \\
    x_{k, m} & \text{if } c=l,j=n \\
    x_{c,j} & \text{otherwise}
\end{cases}$$

\section{Question 2}
\begin{algorithm}
\caption{Sweep x}
\begin{algorithmic}
\STATE Let $x$ be the current configuration
\STATE Let $I$ be the intermediate values\\
\STATE Let $I_c = g(\overline{Al}[x_c], \overline{Fe}[x_c] \overline{Si}[x_c])$
\FOR{k={1...16}; m={1...3}; l={k...17}; n={1..3}}
\STATE $y = y(x, k, l, m, n)$ 
\STATE $d = g(\overline{Al}[y_k], \overline{Fe}[y_k] \overline{Si}[y_k]) + g(\overline{Al}[y_l], \overline{Fe}[y_l] \overline{Si}[y_l]) - I_k - I_l$
\IF{$d > 0$} 
\STATE $x := y$ 
\STATE $I_k = g(\overline{Al}[y_k], \overline{Fe}[y_k] \overline{Si}[y_k])$ 
\STATE $I_l = g(a(\overline{Al}[y_l], \overline{Fe}[y_l] \overline{Si}[y_l]))$ 
\ENDIF
\ENDFOR
\end{algorithmic}
\end{algorithm}

\section{Quesion 4}
Steepest ascent is significantly slower than next ascent and does not 
appear to obtain better maximum values than next ascent. Therefore in
this case it seems next ascent outperforms steepest ascent.

\section{Question 5}
(a) You would expect the problem's objective function to have lots of 
plateus because the objective function is not continuous. This means
there will be lots of cases where two pots are swapped and the
quality, and thus value, will remain constant.

(b) 

\section{Question 6}
$$g''(\overline{Al}, \overline{Fe}, \overline{Si}, x_{c1}, x_{c2}, 
x_{c3}, s) =
\begin{cases}
    g(\overline{Al}, \overline{Fe}, \overline{Si}) - 100000 & \text{if } s_c > s \\
    g(\overline{Al}, \overline{Fe}, \overline{Si}) \text{otherwise}
\end{cases}$$ where
$s_c = \text{max}(x_{c1}, x_{c2}, x_{c3}) - \text{min}(x_{c1}, x_{c2}, x_{c3})$





\end{document}
